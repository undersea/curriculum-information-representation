% Created 2011-01-19 Wed 17:10
\documentclass[11pt]{article}
\usepackage[utf8]{inputenc}
\usepackage[T1]{fontenc}
\usepackage{fixltx2e}
\usepackage{graphicx}
\usepackage{longtable}
\usepackage{float}
\usepackage{wrapfig}
\usepackage{soul}
\usepackage{textcomp}
\usepackage{marvosym}
\usepackage{wasysym}
\usepackage{latexsym}
\usepackage{amssymb}
\usepackage{hyperref}
\tolerance=1000
\providecommand{\alert}[1]{\textbf{#1}}

\title{The management of databases for a large organisation}
\author{James Hurford}
\date{19 January 2011}

\begin{document}

\maketitle

\setcounter{tocdepth}{3}
\tableofcontents
\vspace*{1cm}

\section{Introduction\}}
\label{sec-1}
\section{Workload\}}
\label{sec-2}
\subsection{The Problem\}}
\label{sec-2_1}

Workload is a measure of how many virtual hours each staff member is
going to be doing for the coming year. The purpose of Workload is to
spread the workload as fairly among the staff.  It is a artificial
measurement, using a formula to determine a individuals hour based on
a set of predetermined metrics.

The task is to create a application to take in:


\begin{itemize}
\item For each paper offered
\item For each person teaching this paper how much of the paper they
  are teaching.

\begin{itemize}
\item who the global coordinator is
\item who the local coordinator is
\end{itemize}

\item The Admin roles for each of the members of staff and how much
  the admin roles are worth in Workload hours.
\item For each postgraduate student
\end{itemize}

It could also be used as a tool to indicate where resources are needed
the most, if a extra staff member is needed.
\subsubsection{What it is that this application is replacing\}}
\label{sec-2_1_1}

Currently a Excel spreadsheet is being used to store the input data
and calculate and display the results.  The current solutions is to
complex and needs a easier method for inputting the data.  The amount
of data inputted is coming near to the limit of how much Excel can
take.
\subsubsection{How can a application be more flexible than the spreadsheet solution\}}
\label{sec-2_1_2}
\subsubsection{How to represent it to users?\}}
\label{sec-2_1_3}

What form will this take?  What sort of format will be used?
\subsection{Challenges\}}
\label{sec-2_2}
\subsubsection{What methods of development do I use?\}}
\label{sec-2_2_1}
\subsubsection{What programming platform do I use?\}}
\label{sec-2_2_2}
\subsubsection{Where is it going to be deployed?\}}
\label{sec-2_2_3}
\subsubsection{Security\}}
\label{sec-2_2_4}
\subsubsection{How to represent it to users\}}
\label{sec-2_2_5}
\subsubsection{Where do I get my data from and how do I store it\}}
\label{sec-2_2_6}
\subsubsection{What functionality do I implement\}}
\label{sec-2_2_7}
\subsubsection{Responsiveness of application\}}
\label{sec-2_2_8}
\subsection{Solutions\}}
\label{sec-2_3}
\subsubsection{Method of development\}}
\label{sec-2_3_1}
\subsubsection{Deploying the application\}}
\label{sec-2_3_2}
\subsubsection{Security\}}
\label{sec-2_3_3}
\subsubsection{Framework choice\}}
\label{sec-2_3_4}
\subsubsection{Database and importing of data\}}
\label{sec-2_3_5}
\subsubsection{Feature choice\}}
\label{sec-2_3_6}
\subsubsection{Code optimisation\}}
\label{sec-2_3_7}
\subsection{Conclusion\}}
\label{sec-2_4}
\section{Paper Planner}
\label{sec-3}
\subsection{The Problem\}}
\label{sec-3_1}
\subsubsection{How to represent it to users\}}
\label{sec-3_1_1}
\subsubsection{How to solve it\}}
\label{sec-3_1_2}
\subsection{Solutions chosen\}}
\label{sec-3_2}
\subsubsection{Web Representation\}}
\label{sec-3_2_1}
\subsubsection{Constraints Satisfaction\}}
\label{sec-3_2_2}
\subsubsection{Programming library choice\}}
\label{sec-3_2_3}
\subsubsection{Programming language choice\}}
\label{sec-3_2_4}
\subsection{Challenges\}}
\label{sec-3_3}
\subsubsection{Using python to run C++ code\}}
\label{sec-3_3_1}

Python can import C libraries using ctypes and run C functions.
\subsubsection{How to represent and encode the problem\}}
\label{sec-3_3_2}
\subsection{Conclusion\}}
\label{sec-3_4}

Conclusion is here

\end{document}
